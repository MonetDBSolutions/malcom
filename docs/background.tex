\section{Background}
\label{Background}
In this section we summarize the salient features of resilient intermediates
for query processing., the leading open-source column store and the envisioned
system ExaNest, an exascale HPC platform for data-intensive computing.

\subsection{A Column Store}
%Recap some MonetDB stuff on how queries are compiled.
MonetDB is a widely used column store that internally uses resident
intermediates to break up query processing in well identified steps.
The query plan is broken up into independent steps, glued together into a
dataflow dependency graph. The dataflow graph is greedily consumed by the
database kernel assigning a single core to a single operation.
The resource pressure is kept at a minimal to trim down the degree of parallel
processing when the main resource, RAM, is heavily used.
The system can be instructed to produce an event record for each completed instruction.
This provides a.o. insight into the input/output sizes and timing.

\subsection{Distributed Query Optimizers}
% summarize
The predominant scheme to scale out database processing is to break the base
tables into independent partitions using either a hash-key or ranges.
The partitioning scheme can be used to orchestrate distributed processing
in an optimal way most of the time.

Apache Spark CBO\footnote{\url{https://databricks.com/blog/2017/08/31/cost-based-optimizer-in-apache-spark-2-2.html}}: add some description

\section{Query Execution Model}
In this section we describe how the MonetDB query execution engine was used
 to develop a novel .... We also illustrate the data at our disposal to predict

\subsection{Architecture}
Figure \ref{fig} illustrates the components our reference system, MonetDB,
to execute an sql query. The SQL parser and optimizer layer deploy well-known
rewriting rules to reduce the intermediate sizes and processing time.
It does not rely on any cost-model or pre-computed statistics
(aside from the whereabouts of the data partitions on separate nodes)

The middle layer is a sequence of specialized optimizers that morph the logical
plan received from the SQL compiler into a physical execution plan.
Constant expressions are evaluated, common sub-expressions are identified,
the dataflow graph for parallel processing is derived, etc...

The bottom layer contains the implementation of the relational operators.
Each operator takes as input the resident intermediates produces before
or they access the persistent data on disk.
The actual implementation is often quite complex,
because each operator can be implemented in a multitude of ways.
Since the operator has full knowledge on the actual parameters,
it becomes easy to select the proper path.
Some operators even perform a sampling step before making a choice on the
 preferred algorithm.

To illustrate consider the following simple SQL query
The actual code executed by the MonetDB kernel is shown in Figure \ref{label}.
%clarify what you see

The details of the actual execution can be gathered using the Stethoscope.
A single record is shown in Figure \ref{}. Of interest to this paper are
the properties shown for the arguments and return variables.

\subsection{Profiling Information}
The MonetDB kernel can be instructed to emit profiling events.
Every primitive function comes with an event record taken at the start and
upon completion of the operation. The event record contains details on the
arguments passed, their type and size. Where ever possible the arguments
are linked with the underlying persistent column. Intermediate columns are
nameless and we only can rely on their type/cardinality.
Upon completion, we also know the exact size of the result and the time consumption.
Thread affinity is also available, but for the remainder of this paper ignored.

Explain a MAL instruction
For the remainder of this paper it is necessary to have a basic knowledge of the
MAL programming language in which all SQL queries are translated.
The MAL language is purely designed as intermediate language to express the operations.

The general format is
\begin{verbatim}
content...
\end{verbatim}
Every function belongs to a module. The arguments are either typed scalar
values (:type) or a reference to a column (:bat[:type]).
